\documentclass[11pt]{beamer}
\usetheme{Copenhagen}
\usepackage[utf8]{inputenc}
\usepackage{amsmath}
\usepackage{amsfonts}
\usepackage{amssymb}
\usepackage{verbatim}
%\author{Georgey John}
\setbeamertemplate{headline}{}
\title{Linux Workshop}
%\setbeamercovered{transparent} 
\logo{\includegraphics[height=1.5cm]{logo.png}\hspace{20pt}} 
%\institute{} 
%\date{} 
%\subject{} 
\begin{document}

\begin{frame}
\titlepage
\end{frame}

\begin{frame}{Overview}
\tableofcontents
\end{frame}

\section{What is Linux?}
\begin{frame}{Introduction}
\textbf{\Huge \href{https://www.youtube.com/watch?v=yVpbFMhOAwE}{What is Linux?!!}}\\
\vspace{20pt}
\textbf{\Huge What is this black screen?!}
%Put Word cloud here
\end{frame}

\section{Filesystem}
\begin{frame}{Filesystem}
\begin{center}
\begin{tabular}{|p{3.5cm}|p{3cm}|p{3cm}|}
\hline 
 & \textbf{Windows} & \textbf{Linux} \\ 
\hline 
\textbf{Top directory} & My Computer & / \\ 
\hline 
\textbf{Drives} & C,D,E... & - \\ 
\hline 
\textbf{Separators} & \ & / \\ 
\hline 
\textbf{Case Sensitivity} & No & Yes \\ 
\hline 
\textbf{Delete/Modify open files?} & No & Yes \\ 
\hline 
\textbf{Software/Programs} & Search, download  and install & Direct install from repository \\
\hline 
\textbf{Games} & Game on! & Limited :( \\
\hline 
\end{tabular}
\end{center}
\end{frame}

\section{Basic commands}
\begin{frame}{Basic Linux commands}{Before we dive in}
	\textbf{\large Home directory}\\
	Home directory is the location where you are login. Every user has their own home directory.\\
	\textbf{\large Users}\\
	users are user accounts on a linux system. \\
	\textbf{\large Relative and absolute path}\\
	Relative path is relative to the current directory where you are present.  Absolute path is from the top of the filesystem.\\
	\textbf{\large man page}\\
	man command leads to the manual page of the command.\\

\end{frame}
\begin{frame}{Basic Linux commands}

\begin{columns}
\begin{column}{0.5\textwidth}
\begin{enumerate}
	\item ls
	\item cd
	\item find, locate
	\item pwd
	\item wget 
	\item top, htop
\end{enumerate}
\end{column}
\begin{column}{0.5\textwidth}
   \begin{enumerate}
	\item mkdir
	\item touch
	\item cat
	\item grep
	\item cp
	\item mv
\end{enumerate}
\end{column}
\end{columns}
\end{frame}

\begin{frame}{Basic Linux commands}

\begin{columns}
\begin{column}{0.5\textwidth}
\large \textbf{Navigational and other commands}
\begin{enumerate}
	\item ls \pause - lists all files and directories
	\item cd \pause - changes directory
	\item find \pause - finds file/directory
	\item pwd \pause - present working directory
	\item grep \pause - search in a file
	\item wget \pause - download a file
	\item top \pause - display active processes
\end{enumerate}
\end{column}
\begin{column}{0.5\textwidth}
\large \textbf{File manipulating commands}
   \begin{enumerate}
	\item mkdir \pause - make directory/ies
	\item touch \pause - create new file/s
	\item cat \pause - create/view file/s
	\item cp \pause - copy file(s) /directory(ies)
	\item mv \pause - move (rename) file(s) /directory(ies)
\end{enumerate}
\end{column}
\end{columns}
\end{frame}

\subsection{commands syntax}
\begin{frame}{Linux command syntax}
\large \$ command [-argument] [-argument] [--argument] [file]\vspace{1cm}\pause \\
Variations exist, but essence is same\vspace{4cm}
 
Example :\\
Which commands have you used so far?
\end{frame}

\subsection{Navigational commands}
\begin{frame}{ls}
\textbf{Purpose}:\\
\large lists directory contents \vspace{1cm}\pause \\
 
\textbf{Common usages}:\\
\begin{enumerate}
\item ls : List all files in the current directory
\item ls -l : List line in long format;one per line with details
\item ls -la : List as previous, but also hidden files
\item ls path/to/directory : Lists files on the specified directory
\end{enumerate}
\end{frame}

\begin{frame}{cd}
\textbf{Purpose}:\\
\large change directory \vspace{1cm}\pause \\
 
\textbf{Common usages}:\\
\begin{enumerate}
\item cd and cd $\scriptstyle\mathtt{\sim}$ : change to home directory
\item cd path/to/directory : change directory to specified one
\item cd .. : change to the parent directory 
\item cd - : return to previous directory
\end{enumerate}
\end{frame}

\begin{frame}{find}
\textbf{Purpose}:\\
\large find files by name \vspace{1cm}\pause \\
 
\textbf{Common usages}:\\
\begin{enumerate}
\item find /path/to/directory -name filename : search for files in a directory hierarchy
\item find /path/to/directory/ -mtime 7 : finds all files modified in last 7 days in given directory
\item find /path/to/directory/ -iname ".py" : finds .py files in given directory
\end{enumerate}
\end{frame}

\begin{frame}{pwd}
\textbf{Purpose}:\\
\large displays current working directory\vspace{1cm}\pause \\
 
\textbf{Common usages}:\\
\begin{enumerate}
\item pwd : prints the current working directory on the screen
\end{enumerate}
\end{frame}

\begin{frame}{grep}
\textbf{Purpose}:\\
\large search for a specific string in a specific file\vspace{1cm}\pause \\
 
\textbf{Common usages}:\\
\begin{enumerate}
\item grep "string" filename : search for "string" in the file
\item grep -i "string" filename : case insensitive search for "string" in the file
\item grep -A N "string" FILENAME :prints the specified N lines after the match.
\end{enumerate}
\end{frame}

\begin{frame}{wget}
\textbf{Purpose}:\\
\large download a file from a link\vspace{1cm}\pause \\
 
\textbf{Common usages}:\\
\begin{enumerate}
\item wget url : search for "string" in the file
\end{enumerate}
\end{frame}

\begin{frame}{top/htop}
\textbf{Purpose}:\\
\large displays all active processes\vspace{1cm}\pause \\
 
\textbf{Common usages}:\\
\begin{enumerate}
\item top : displays all active processes
\item top -u username : displays all active processes run by username
\end{enumerate}
\end{frame}

\subsection{File manipulating commands}
\begin{frame}{mkdir}
\textbf{Purpose}:\\
\large Creates a directory\vspace{1cm}\pause \\
 
\textbf{Common usages}:\\
\begin{enumerate}
\item mkdir : Creates a directory
\end{enumerate}
\end{frame}

\begin{frame}{touch}
\textbf{Purpose}:\\
\large Creates file/s\vspace{1cm}\pause \\
 
\textbf{Common usages}:\\
\begin{enumerate}
\item touch filename: Creates an empty file called filename
\end{enumerate}
\end{frame}

\begin{frame}{cat}
\textbf{Purpose}:\\
\large 1. Creates an empty file/s if file does not exist\\
2.  view contain of file if file exist\vspace{1cm}\pause \\
 
\textbf{Common usages}:\\
\begin{enumerate}
\item cat filename: displays content of file
\item cat $>$filename : creates file and takes input until Ctrl+d
\item cat -n filename: displays content of file with line number

\end{enumerate}
\end{frame}

\begin{frame}{cp}
\textbf{Purpose}:\\
\large copies file/s from source directory to destination directory\vspace{1cm}\pause \\
 
\textbf{Common usages}:\\
\begin{enumerate}
\item cp source destination: copy file/s from source directory to destination directory
\item cp -R source destination : copy file/s recursively from source directory to destination directory
\end{enumerate}
\end{frame}

\begin{frame}{mv}
\textbf{Purpose}:\\
\large moves file/s from source directory to destination directory\vspace{1cm}\pause \\
 
\textbf{Common usages}:\\
\begin{enumerate}
\item mv source destination: move file/s from source directory to destination directory
\end{enumerate}
\end{frame}

\section{Wildcards, pipes, redirection and shortcuts}
\begin{frame}{Wildcards}
A wildcard is a character that can be used as a substitute for any of a class of characters in a search, thereby greatly increasing the flexibility and efficiency of searches.\\
It can be used with  mv, cp, rm, ls and other commands.\\ \vspace{1cm}
\textbf{\large Common wildcards}
\begin{enumerate}
\item * : represents any combination of characters.\pause
\item ? : represents a single character.\pause
\item {[ ]} : range of characters mentioned inside the square bracket.
\end{enumerate}
\end{frame}

\begin{frame}{pipes}
A pipe command ($|$) sends output of first command as input of second command.\\ \vspace{1cm}
\textbf{\large Common usages}
\begin{enumerate}
\item ls -l $|$ less
\item man command $|$ grep "string"
\item cat filename $|$ head -20
\end{enumerate}
\end{frame}

\begin{frame}{redirection}
A redirection operator sends the output of a command to a file\\ \vspace{1cm}
\textbf{\large redirection operators}
\begin{enumerate}
\item \textgreater  : redirects the output to a file, clears contents if file exists
\item \textgreater \textgreater : redirects the output to a new file, appends to file if it exists.
\item \textless : sends data from a file as an input to command
\end{enumerate}
\end{frame}

\begin{frame}{Shortcuts}
\begin{enumerate}
\item Tab : for auto completion
\item Ctrl+l : Clear screen
\item Ctrl+c : kill a process
\item Ctrl+d : logout/exit
\item Select text and Ctrl+Shift+c : copy text
\item Ctrl+Shift+v : paste text
\item Ctrl+r : reverse search
\end{enumerate}
\end{frame}

\section{Basic Server operations}
\begin{frame}{SSH}
ssh or secure shell is command used to login onto a remove server and executing commands.\\ \vspace{1cm}
\textbf{Common usages}:\\
\begin{enumerate}
\item ssh username@ipaddress : logs onto username on the system having ipaddress
\item ssh -X username@ipaddress : logs onto username on the system having ipaddress and uses graphics.
\end{enumerate}
\end{frame}

\begin{frame}{SCP}
scp or secure copy command copies files from origin to destination\\ \vspace{1cm}
\textbf{Common usages}:\\
\begin{enumerate}
\item scp username@ipaddress:/path/to/file /path/on/your/laptop : copies file from username on the system having ipaddress to your laptop.
\item scp /path/on/your/laptop username@ipaddress:/path/to/file : copies file from your laptop to username on the system having ipaddress.
\end{enumerate}
\end{frame}

\begin{frame}{screen}
opens a new tab for terminal\\ \vspace{1cm}
\textbf{Common usages}:\\
\begin{enumerate}
\item screen : opens a new screen

\item screen -list : lists active minimized screen
\item screen -r screenid : reopens the minimized screen
\item Ctrl+d : minimizes the screen and detach screen (return back to
terminal)
\end{enumerate}
\end{frame}

\begin{frame}{\&}
frees the prompt\\ \vspace{1cm}
\textbf{Common usages}:\\
\begin{enumerate}
\item python demo.py \& :  runs python but prompt is free
\item python demo.py 2\textgreater \&1 \textbackslash dev\textbackslash null \& : runs python but prompt is free and no output is shown.
\end{enumerate}
\end{frame}

\begin{frame}{Ctrl+z}
suspends the program\\ \vspace{1cm}
\textbf{Common usages}:\\
\begin{enumerate}
\item Ctrl+z and then bg :  Ctrl+z suspends the program and bg sends the program to background. 
\item fg : fg brings back the most recently suspended program to foreground
\end{enumerate}
\end{frame}

\begin{frame}{}
\Huge Thank you!! \\
\large Suggestions are welcome!
\end{frame}
\end{document}
